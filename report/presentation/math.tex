% Sam
% - maybe MPQCs?

% \begin{frame}
%   \frametitle{Density matrices recap}

%   In a pure state $\ket\psi$, the expected value of an observable $A$ is:

%   \[ \expected{A}_\psi = \braopket{\psi}{A}{\psi} \]

%   The probability of measuring the system in some basis can be found by using an
%   orthogonal projector:

%   \[ \Pi_i = \proj{\phi_i} \]

%   In a mixed state $\rho$, the expected value looks like:

%   \[ \expected{A}_\psi = \Tr(A \rho) = \sum_i \braopket{i}{A \rho}{i} \]

%   (Trace is independent of basis---can sum over any basis $\{\ket{i}\}$)

% \end{frame}

\begin{frame}
  \frametitle{Partial trace}

  In a mixed state $\rho$, the expected value looks like:

  \[ \expected{A}_\psi = \Tr(A \rho) = \sum_i \braopket{i}{A \rho}{i} \]

  If a measurement traces over a complete set of basis states for the space
  $\mathcal{H}$, what if we want to discard the subsystem $\mathcal{A}$ of
  $\mathcal{H}_{\mathcal{A}} \otimes \mathcal{H}_{\mathcal{B}}$?

  \[ \Tr_{\mathcal{A}}(\rho) =
  \sum_i (\bra{i}_{\mathcal{A}} \otimes I_{\mathcal{B}})
  \ \rho\ (\ket{i}_{\mathcal{A}} \otimes I_{\mathcal{B}})\]

  By \emph{tracing out} the system $\mathcal{A}$, we project onto a basis for
  $\mathcal{A}$ while leaving $\mathcal{B}$ untouched ($I_{\mathcal{B}}$).

  Each term of the sum leaves us with an \emph{operator} instead of a scalar.
  This is the \textbf{partial trace}\autocite{quic06}.

\end{frame}

\begin{frame}
  \frametitle{Partial measurement}

  Let's try the partial trace:

 \[\Tr_{\mathcal{A}}((\Pi_i \otimes I_{\mathcal{B}}) \rho)
    = \sum_j (\bra{j} \otimes I) (\Pi_i \otimes I) \rho (\ket{j} \otimes I)\]

  Notice that $\bra{j} \Pi_j$ is $0$ if $i\neq j$, and $\bra{j}$ otherwise!

  \[ \rho' = \frac{\Tr_{\mathcal{A}}((\Pi_i\otimes I_{\mathcal{B}}) \rho)}
    {\LaTeXunderbrace{ \Tr((\Pi_i\otimes I_{\mathcal{B}})\rho) }_{\text{magic}}}\]

  Making a partial measurement on our system transforms the density matrix
  \emph{non-linearly}!

\end{frame}

\begin{frame}
    \frametitle{Multilayer parameterized quantum circuits}

    \begin{center}
        % \begin{quantikz}
        % \lstick{$\mathcal{A} \ket{0}$} & \gate{R} & \gate[wires=2]{$U_E$} & \qw \\
        % \lstick{$\ket{0}$} & \gate{R} &  & \qw \\
        % \lstick{$\ket{0}$} & \gate{R} &  & \qw \\
        % \end{quantikz}
    \end{center}
\end{frame}
