\section{Introduction}
In recent years, Generative Adversarial Networks (GANs) have revolutionized deep learning by enabling the generation of realistic images, videos, and audio. While classical GANs have been extensively studied, quantum GANs have emerged as a promising alternative due to their potential to achieve superior performance using fewer resources. The research paper titled "Experimental Quantum Generative Adversarial Networks for Image Generation" proposes a new method for image generation using GANs trained on a quantum computer. The paper delves into the theory behind GANs, discussing why they are expected to converge and how loss is computed. It also compares the benefits of quantum GANs to classical GANs and presents the results obtained using a batch strategy. Additionally, the authors ran experiments on a real superconducting quantum processor provided by IBM Quantum, which demonstrates the potential of quantum GANs. In this report, we present our implementation of the paper, aiming to reproduce and extend the key results presented while providing additional insights and contributions. We follow a similar structure to the original paper and discuss the theory and derivation of key results, implementation details, and experimental results.
