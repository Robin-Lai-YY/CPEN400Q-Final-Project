\section{Requirements}

\subsection{Written report}
The target audience of this report is the professor. It should contain an in-depth discussion of the theory and derivation of key results, and the tables, plots, and other relevant results generated by your software implementation that reproduce the original work. It should follow loosely the same structure of the original paper, but be expressed in your own words.

It should also contain:
\begin{itemize}
    \item rationale for choice of programming framework
    \item a discussion of any issues encountered that affected reproducibility
\end{itemize}

Further to this second point, reproducing the work of others is extremely challenging. It is common for papers to, e.g., omit simulation parameters, make unstated assumptions, or even include outright errors. As such, you will not be graded on how close your results are, but rather, how close your results are to what the paper intended to show, as a function of the information that was available to you. A critical analysis of how reproducible the work is, and what your group's thought process was for handling ambiguities, is therefore an important component of the report. My hope is that this experience will allow you to think more carefully about how you present your own work in the future, and to promote open and reproducible scientific practices.
